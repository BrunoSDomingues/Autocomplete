\documentclass[a4paper,11pt,final]{article}
\usepackage{fancyvrb, color, graphicx, hyperref, amsmath, url}
\usepackage{palatino}
\usepackage{pygments}
\usepackage[T1]{fontenc}
\usepackage[a4paper,text={16.5cm,25.2cm},centering]{geometry}

\hypersetup  
{   pdfauthor = {Bruno Domingues e Michel Moraes},
  pdftitle={Autocomplete},
  colorlinks=TRUE,
  linkcolor=black,
  citecolor=blue,
  urlcolor=blue
}

\setlength{\parindent}{0pt}
\setlength{\parskip}{1.2ex}


        
\title{Projeto Final de NLP - Autocomplete}
\author{Bruno Domingues e Michel Moraes}

\begin{document}
\maketitle

\section{Abstract}

Nos dias de hoje, quase todos os textos que escrevemos no cotidiano, são digitados em alguma plataforma virtual, sendo aplicativos de mensagem texto, e redes sociais, o exemplo mais comum, já que uma grande parte das pessoas manda mensagens de texto durante todo o dia, tanto para assuntos pessoais, ou profissionais, quanto como um meio de entretenimento. Relatórios e documentos extensos muitas vezes são redigidos em um computador, ou dispositivo móvel, também.

Uma ferramenta que pode agilizar bastante na hora de escrever um texto é sugestor automático de palavras, comumente visto em sites como o Google, e em teclados de dispositivos móveis. Desta forma, neste relatório se encontra uma análise de um dataset de frases, e buscamos responder se é possível fazer um sugestor automático simples com algumas ferramentas de Processamento Linguagem Natural com Python.

Palavras-chave: Processamento de linguagem natural. Autosuggest. Autocomplete. N-gramas.


\section{Introdução}

O uso de abreviações e atalhos na hora de digitar se encontra cada vez mais presente no dia a dia: seja ao mandar mensagens, escrever textos ou até mesmo para pesquisar na internet.

A partir deste desejo de economizar tempo, surgiram diversas ferramentas capazes de completar frases e palavras, conhecidas geralmente como autocomplete e autosuggest.

O autosuggest pode não só economizar tempo na hora de escrever grandes textos, ou rápidas mensagens, mas também diminui a quantidade de erros ortográficos, já que um bom autosuggest vai acertar nas sugestões de palavras na maioria das vezes, e essas não possuem erros ortográficos, ou de digitação.

Tendo em vista as várias vantagens deste tipo de ferramenta, optou-se por desenvolvê-la, utilizando um dataset e ferramentas de Python tanto para processar os dados quanto para montar a ferramenta.

\section{Dados e pré-processamento}

O primeiro passo para tentar construir um autocomplete é reunir um grande conjunto de frases. Para atingir tal objetivo, utilizou-se um dataset [1] que contém artigos de jornais diversos e textos de blogs.
O dataset possui fontes de diversas línguas, e para este desenvolvimento optou-se por utilizar a língua inglesa.

Além do texto, o dataset disponibiliza também a data de publicação e a fonte do texto (geralmente um link para o artigo/blog); no entanto, estas informações não são relevantes, e podem ser descartadas.

\subsection{Criando o dataframe inicial}

O código abaixo mostra o processo de importar o dataset e filtrar por língua. Visto que o arquivo original era muito grande, utilizou-se a biblioteca Dask para processar o arquivo, uma vez que ela possui ferramentas e estrutura dedicada para tal.



\begin{Verbatim}[commandchars=\\\{\},frame=single,fontsize=\small, xleftmargin=0.5em]
\PY{c+c1}{\PYZsh{} Imports utilizados}
\PY{k+kn}{import} \PY{n+nn}{dask}\PY{n+nn}{.}\PY{n+nn}{dataframe} \PY{k}{as} \PY{n+nn}{dd}
\PY{k+kn}{import} \PY{n+nn}{pandas} \PY{k}{as} \PY{n+nn}{pd}
\PY{k+kn}{import} \PY{n+nn}{glob}\PY{o}{,} \PY{n+nn}{re}\PY{o}{,} \PY{n+nn}{os}

\PY{c+c1}{\PYZsh{} Path do dataset}
\PY{n}{cwd} \PY{o}{=} \PY{n}{os}\PY{o}{.}\PY{n}{getcwd}\PY{p}{(}\PY{p}{)}\PY{o}{.}\PY{n}{replace}\PY{p}{(}\PY{l+s+sa}{r}\PY{l+s+s2}{\PYZdq{}}\PY{l+s+s2}{/relatorio}\PY{l+s+s2}{\PYZdq{}}\PY{p}{,} \PY{l+s+s2}{\PYZdq{}}\PY{l+s+s2}{\PYZdq{}}\PY{p}{)}
\PY{n}{path} \PY{o}{=} \PY{n}{os}\PY{o}{.}\PY{n}{path}\PY{o}{.}\PY{n}{join}\PY{p}{(}\PY{n}{cwd}\PY{p}{,} \PY{l+s+s2}{\PYZdq{}}\PY{l+s+s2}{data}\PY{l+s+s2}{\PYZdq{}}\PY{p}{)}
\PY{n}{filepath} \PY{o}{=} \PY{n}{os}\PY{o}{.}\PY{n}{path}\PY{o}{.}\PY{n}{join}\PY{p}{(}\PY{n}{path}\PY{p}{,} \PY{l+s+s2}{\PYZdq{}}\PY{l+s+s2}{old\PYZhy{}newspaper.tsv}\PY{l+s+s2}{\PYZdq{}}\PY{p}{)}

\PY{c+c1}{\PYZsh{} Lendo os dados do arquivo tsv}
\PY{n}{ddf} \PY{o}{=} \PY{n}{dd}\PY{o}{.}\PY{n}{read\PYZus{}csv}\PY{p}{(}\PY{n}{filepath}\PY{p}{,} \PY{n}{sep}\PY{o}{=}\PY{l+s+s2}{\PYZdq{}}\PY{l+s+se}{\PYZbs{}t}\PY{l+s+s2}{\PYZdq{}}\PY{p}{)}

\PY{c+c1}{\PYZsh{} Selecionando somente em inglês e somente a coluna Text}
\PY{c+c1}{\PYZsh{} pois os outros dados não serão utilizados}
\PY{n}{df} \PY{o}{=} \PY{n}{ddf}\PY{p}{[}\PY{n}{ddf}\PY{p}{[}\PY{l+s+s2}{\PYZdq{}}\PY{l+s+s2}{Language}\PY{l+s+s2}{\PYZdq{}}\PY{p}{]} \PY{o}{==} \PY{l+s+s2}{\PYZdq{}}\PY{l+s+s2}{English}\PY{l+s+s2}{\PYZdq{}}\PY{p}{]}
\PY{n}{df} \PY{o}{=} \PY{n}{df}\PY{p}{[}\PY{p}{[}\PY{l+s+s2}{\PYZdq{}}\PY{l+s+s2}{Text}\PY{l+s+s2}{\PYZdq{}}\PY{p}{]}\PY{p}{]}
\end{Verbatim}


A biblioteca Dask funciona separando o arquivo em partições diversas, o que facilita o processamento. No entanto, ao filtrar por língua foram geradas partições vazias.
Como elas não serão utilizadas, elas foram removidas. Após sua remoção, converteu-se o dataframe do Dask para um dataframe do Pandas, visto que algumas chamadas de função não são as mesmas.



\begin{Verbatim}[commandchars=\\\{\},frame=single,fontsize=\small, xleftmargin=0.5em]
\PY{c+c1}{\PYZsh{} Limpeza de particoes vazias}
\PY{c+c1}{\PYZsh{} https://stackoverflow.com/questions/47812785/remove\PYZhy{}empty\PYZhy{}partitions\PYZhy{}in\PYZhy{}dask}
\PY{k}{def} \PY{n+nf}{cull\PYZus{}empty\PYZus{}partitions}\PY{p}{(}\PY{n}{df}\PY{p}{)}\PY{p}{:}
    \PY{n}{ll} \PY{o}{=} \PY{n+nb}{list}\PY{p}{(}\PY{n}{df}\PY{o}{.}\PY{n}{map\PYZus{}partitions}\PY{p}{(}\PY{n+nb}{len}\PY{p}{)}\PY{o}{.}\PY{n}{compute}\PY{p}{(}\PY{p}{)}\PY{p}{)}
    \PY{n}{df\PYZus{}delayed} \PY{o}{=} \PY{n}{df}\PY{o}{.}\PY{n}{to\PYZus{}delayed}\PY{p}{(}\PY{p}{)}
    \PY{n}{df\PYZus{}delayed\PYZus{}new} \PY{o}{=} \PY{n+nb}{list}\PY{p}{(}\PY{p}{)}
    \PY{n}{pempty} \PY{o}{=} \PY{k+kc}{None}
    
    \PY{k}{for} \PY{n}{ix}\PY{p}{,} \PY{n}{n} \PY{o+ow}{in} \PY{n+nb}{enumerate}\PY{p}{(}\PY{n}{ll}\PY{p}{)}\PY{p}{:}
        \PY{k}{if} \PY{l+m+mi}{0} \PY{o}{==} \PY{n}{n}\PY{p}{:}
            \PY{n}{pempty} \PY{o}{=} \PY{n}{df}\PY{o}{.}\PY{n}{get\PYZus{}partition}\PY{p}{(}\PY{n}{ix}\PY{p}{)}
            
        \PY{k}{else}\PY{p}{:}
            \PY{n}{df\PYZus{}delayed\PYZus{}new}\PY{o}{.}\PY{n}{append}\PY{p}{(}\PY{n}{df\PYZus{}delayed}\PY{p}{[}\PY{n}{ix}\PY{p}{]}\PY{p}{)}
            
    \PY{k}{if} \PY{n}{pempty} \PY{o+ow}{is} \PY{o+ow}{not} \PY{k+kc}{None}\PY{p}{:}
        \PY{n}{df} \PY{o}{=} \PY{n}{dd}\PY{o}{.}\PY{n}{from\PYZus{}delayed}\PY{p}{(}\PY{n}{df\PYZus{}delayed\PYZus{}new}\PY{p}{,} \PY{n}{meta}\PY{o}{=}\PY{n}{pempty}\PY{p}{)}
        
    \PY{k}{return} \PY{n}{df}

\PY{n}{df} \PY{o}{=} \PY{n}{cull\PYZus{}empty\PYZus{}partitions}\PY{p}{(}\PY{n}{df}\PY{p}{)}

\PY{c+c1}{\PYZsh{} Convertendo em um dataframe do Pandas novamente}
\PY{n}{data} \PY{o}{=} \PY{n}{df}\PY{o}{.}\PY{n}{compute}\PY{p}{(}\PY{p}{)}
\end{Verbatim}


O dataframe possui pouco mais de 1 milhão de artigos e blogs distintos, o que é suficiente para a análise.

\subsection{Limpeza e processamento dos dados}

Para poder criar a ferramenta, deve-se primeiro fazer uma limpeza no dataframe, visto que há presença de caracteres especiais, URLs e outros que podem atrapalhar a predição.

Primeiro, removeu-se todas as URLs presentes no texto.



\begin{Verbatim}[commandchars=\\\{\},frame=single,fontsize=\small, xleftmargin=0.5em]
\PY{c+c1}{\PYZsh{} Removendo URLs}
\PY{k}{def} \PY{n+nf}{limpa\PYZus{}url}\PY{p}{(}\PY{n}{texto}\PY{p}{)}\PY{p}{:}
    \PY{c+c1}{\PYZsh{} Regex obtida de https://www.geeksforgeeks.org/python\PYZhy{}check\PYZhy{}url\PYZhy{}string/}
    \PY{n}{pattern} \PY{o}{=} \PY{l+s+sa}{r}\PY{l+s+s2}{\PYZdq{}\PYZdq{}\PYZdq{}}
\PY{l+s+s2}{        (?i)  \PYZsh{} Ignore case.}
\PY{l+s+s2}{        }\PY{l+s+s2}{\PYZbs{}}\PY{l+s+s2}{b  \PYZsh{} Inicio de palavra.}
\PY{l+s+s2}{        (?:}
\PY{l+s+s2}{            https?://}
\PY{l+s+s2}{        |}
\PY{l+s+s2}{            www}
\PY{l+s+s2}{            }\PY{l+s+s2}{\PYZbs{}}\PY{l+s+s2}{d}\PY{l+s+s2}{\PYZob{}}\PY{l+s+s2}{0,3\PYZcb{}}
\PY{l+s+s2}{            [.]}
\PY{l+s+s2}{        |}
\PY{l+s+s2}{            [a\PYZhy{}z0\PYZhy{}9.}\PY{l+s+s2}{\PYZbs{}}\PY{l+s+s2}{\PYZhy{}]+}
\PY{l+s+s2}{            [.]}
\PY{l+s+s2}{            [a\PYZhy{}z]}\PY{l+s+s2}{\PYZob{}}\PY{l+s+s2}{2,4\PYZcb{}}
\PY{l+s+s2}{            /}
\PY{l+s+s2}{        )}
\PY{l+s+s2}{        (?:}
\PY{l+s+s2}{            [\PYZca{}}\PY{l+s+s2}{\PYZbs{}}\PY{l+s+s2}{s()\PYZlt{}\PYZgt{}]+}
\PY{l+s+s2}{        |}
\PY{l+s+s2}{            }\PY{l+s+s2}{\PYZbs{}}\PY{l+s+s2}{(}
\PY{l+s+s2}{            (?:}
\PY{l+s+s2}{                [\PYZca{}}\PY{l+s+s2}{\PYZbs{}}\PY{l+s+s2}{s()\PYZlt{}\PYZgt{}]+}
\PY{l+s+s2}{            |}
\PY{l+s+s2}{                }\PY{l+s+s2}{\PYZbs{}}\PY{l+s+s2}{(}
\PY{l+s+s2}{                [\PYZca{}}\PY{l+s+s2}{\PYZbs{}}\PY{l+s+s2}{s()\PYZlt{}\PYZgt{}]+}
\PY{l+s+s2}{                }\PY{l+s+s2}{\PYZbs{}}\PY{l+s+s2}{)}
\PY{l+s+s2}{            )*}
\PY{l+s+s2}{            }\PY{l+s+s2}{\PYZbs{}}\PY{l+s+s2}{)}
\PY{l+s+s2}{        )+}
\PY{l+s+s2}{        (?:}
\PY{l+s+s2}{            }\PY{l+s+s2}{\PYZbs{}}\PY{l+s+s2}{(}
\PY{l+s+s2}{            (?:}
\PY{l+s+s2}{                [\PYZca{}}\PY{l+s+s2}{\PYZbs{}}\PY{l+s+s2}{s()\PYZlt{}\PYZgt{}]+}
\PY{l+s+s2}{            |}
\PY{l+s+s2}{                }\PY{l+s+s2}{\PYZbs{}}\PY{l+s+s2}{(}
\PY{l+s+s2}{                [\PYZca{}}\PY{l+s+s2}{\PYZbs{}}\PY{l+s+s2}{s()\PYZlt{}\PYZgt{}]+}
\PY{l+s+s2}{                }\PY{l+s+s2}{\PYZbs{}}\PY{l+s+s2}{)}
\PY{l+s+s2}{            )*}
\PY{l+s+s2}{            }\PY{l+s+s2}{\PYZbs{}}\PY{l+s+s2}{)}
\PY{l+s+s2}{        |}
\PY{l+s+s2}{            [\PYZca{}}\PY{l+s+s2}{\PYZbs{}}\PY{l+s+s2}{s`!()}\PY{l+s+s2}{\PYZbs{}}\PY{l+s+s2}{[}\PY{l+s+s2}{\PYZbs{}}\PY{l+s+s2}{]}\PY{l+s+si}{\PYZob{}\PYZcb{}}\PY{l+s+s2}{;:}\PY{l+s+s2}{\PYZsq{}}\PY{l+s+s2}{\PYZbs{}}\PY{l+s+s2}{\PYZdq{}}\PY{l+s+s2}{.,\PYZlt{}\PYZgt{}?«»“”‘’]}
\PY{l+s+s2}{        )}
\PY{l+s+s2}{    }\PY{l+s+s2}{\PYZdq{}\PYZdq{}\PYZdq{}}
    \PY{n}{repl} \PY{o}{=} \PY{l+s+s2}{\PYZdq{}}\PY{l+s+s2}{\PYZdq{}}
    \PY{n}{matcher} \PY{o}{=} \PY{n}{re}\PY{o}{.}\PY{n}{compile}\PY{p}{(}\PY{n}{pattern}\PY{p}{,} \PY{n}{re}\PY{o}{.}\PY{n}{VERBOSE}\PY{p}{)}
    \PY{k}{return} \PY{n}{matcher}\PY{o}{.}\PY{n}{sub}\PY{p}{(}\PY{n}{repl}\PY{p}{,} \PY{n}{texto}\PY{p}{)}

\PY{n}{data}\PY{p}{[}\PY{l+s+s2}{\PYZdq{}}\PY{l+s+s2}{Text}\PY{l+s+s2}{\PYZdq{}}\PY{p}{]} \PY{o}{=} \PY{n}{data}\PY{p}{[}\PY{l+s+s2}{\PYZdq{}}\PY{l+s+s2}{Text}\PY{l+s+s2}{\PYZdq{}}\PY{p}{]}\PY{o}{.}\PY{n}{apply}\PY{p}{(}\PY{k}{lambda} \PY{n}{t}\PY{p}{:} \PY{n}{limpa\PYZus{}url}\PY{p}{(}\PY{n}{t}\PY{p}{)}\PY{p}{)}
\end{Verbatim}


Em seguida, removeu-se caracteres especiais. Vale notar que caracteres como pontos de interrogação e de exclamação devem ser conservados, pois ainda indicam o fim de uma frase, somente mudando sua entonação.
Para conservá-los, substituiu-se estes caracteres pelo ponto final, visto que a entonação da frase não é o objetivo desta análise.



\begin{Verbatim}[commandchars=\\\{\},frame=single,fontsize=\small, xleftmargin=0.5em]
\PY{c+c1}{\PYZsh{} Removendo caracteres especiais \PYZdq{}@, \PYZsh{}, \PYZdl{}, \PYZpc{}, \PYZam{}, *, `\PYZdq{}}
\PY{n}{special} \PY{o}{=} \PY{l+s+sa}{r}\PY{l+s+s2}{\PYZdq{}}\PY{l+s+s2}{[}\PY{l+s+s2}{\PYZbs{}}\PY{l+s+s2}{@}\PY{l+s+s2}{\PYZbs{}}\PY{l+s+s2}{\PYZsh{}}\PY{l+s+s2}{\PYZbs{}}\PY{l+s+s2}{\PYZdl{}}\PY{l+s+s2}{\PYZbs{}}\PY{l+s+s2}{\PYZpc{}}\PY{l+s+s2}{\PYZbs{}}\PY{l+s+s2}{\PYZam{}}\PY{l+s+s2}{\PYZbs{}}\PY{l+s+s2}{*}\PY{l+s+s2}{\PYZbs{}}\PY{l+s+s2}{`]}\PY{l+s+s2}{\PYZdq{}}
\PY{n}{data}\PY{p}{[}\PY{l+s+s2}{\PYZdq{}}\PY{l+s+s2}{Text}\PY{l+s+s2}{\PYZdq{}}\PY{p}{]} \PY{o}{=} \PY{n}{data}\PY{p}{[}\PY{l+s+s2}{\PYZdq{}}\PY{l+s+s2}{Text}\PY{l+s+s2}{\PYZdq{}}\PY{p}{]}\PY{o}{.}\PY{n}{apply}\PY{p}{(}\PY{k}{lambda} \PY{n}{t}\PY{p}{:} \PY{n}{re}\PY{o}{.}\PY{n}{sub}\PY{p}{(}\PY{n}{special}\PY{p}{,} \PY{l+s+s2}{\PYZdq{}}\PY{l+s+s2}{\PYZdq{}}\PY{p}{,} \PY{n}{t}\PY{p}{)}\PY{p}{)}

\PY{c+c1}{\PYZsh{} Fazendo replace de ? e ! por .}
\PY{n}{data}\PY{p}{[}\PY{l+s+s2}{\PYZdq{}}\PY{l+s+s2}{Text}\PY{l+s+s2}{\PYZdq{}}\PY{p}{]} \PY{o}{=} \PY{n}{data}\PY{p}{[}\PY{l+s+s2}{\PYZdq{}}\PY{l+s+s2}{Text}\PY{l+s+s2}{\PYZdq{}}\PY{p}{]}\PY{o}{.}\PY{n}{apply}\PY{p}{(}\PY{k}{lambda} \PY{n}{t}\PY{p}{:} \PY{n}{t}\PY{o}{.}\PY{n}{replace}\PY{p}{(}\PY{l+s+s2}{\PYZdq{}}\PY{l+s+s2}{?}\PY{l+s+s2}{\PYZdq{}}\PY{p}{,} \PY{l+s+s2}{\PYZdq{}}\PY{l+s+s2}{.}\PY{l+s+s2}{\PYZdq{}}\PY{p}{)}\PY{p}{)}
\PY{n}{data}\PY{p}{[}\PY{l+s+s2}{\PYZdq{}}\PY{l+s+s2}{Text}\PY{l+s+s2}{\PYZdq{}}\PY{p}{]} \PY{o}{=} \PY{n}{data}\PY{p}{[}\PY{l+s+s2}{\PYZdq{}}\PY{l+s+s2}{Text}\PY{l+s+s2}{\PYZdq{}}\PY{p}{]}\PY{o}{.}\PY{n}{apply}\PY{p}{(}\PY{k}{lambda} \PY{n}{t}\PY{p}{:} \PY{n}{t}\PY{o}{.}\PY{n}{replace}\PY{p}{(}\PY{l+s+s2}{\PYZdq{}}\PY{l+s+s2}{!}\PY{l+s+s2}{\PYZdq{}}\PY{p}{,} \PY{l+s+s2}{\PYZdq{}}\PY{l+s+s2}{.}\PY{l+s+s2}{\PYZdq{}}\PY{p}{)}\PY{p}{)}
\end{Verbatim}


Em seguida, removeu-se todos os caracteres que não fossem letras, espaços, pontos finais e apóstrofes.
As apóstrofes foram conservadas pois são muito utilizadas para junção de duas palavras em uma só na língua inglesa (como "don't" e "I'm").

No entanto, na hora de tokenizar estas palavras, o tokenizador as separa de forma inadequada ("don't" -> "do", "n't").
Assim, substituiu-se as apóstrofes pelo hífen, pois assim o tokenizador interpreta como uma palavra só. 
Esta substituição foi realizada após a remoção dos caracteres, de modo a garantir que não houvessem hífens novos que pudessem gerar problemas na hora de substituir o hífen de volta pela apóstrofe na predição.



\begin{Verbatim}[commandchars=\\\{\},frame=single,fontsize=\small, xleftmargin=0.5em]
\PY{c+c1}{\PYZsh{} Removendo pontuações diversas exceto apostrofes e pontos finais}
\PY{n}{data}\PY{p}{[}\PY{l+s+s2}{\PYZdq{}}\PY{l+s+s2}{Text}\PY{l+s+s2}{\PYZdq{}}\PY{p}{]} \PY{o}{=} \PY{n}{data}\PY{p}{[}\PY{l+s+s2}{\PYZdq{}}\PY{l+s+s2}{Text}\PY{l+s+s2}{\PYZdq{}}\PY{p}{]}\PY{o}{.}\PY{n}{apply}\PY{p}{(}\PY{k}{lambda} \PY{n}{t}\PY{p}{:} \PY{n}{re}\PY{o}{.}\PY{n}{sub}\PY{p}{(}\PY{l+s+sa}{r}\PY{l+s+s2}{\PYZdq{}}\PY{l+s+s2}{[\PYZca{}}\PY{l+s+s2}{\PYZbs{}}\PY{l+s+s2}{w}\PY{l+s+s2}{\PYZbs{}}\PY{l+s+s2}{s}\PY{l+s+s2}{\PYZbs{}}\PY{l+s+s2}{\PYZsq{}}\PY{l+s+s2}{\PYZbs{}}\PY{l+s+s2}{.]}\PY{l+s+s2}{\PYZdq{}}\PY{p}{,} \PY{l+s+s2}{\PYZdq{}}\PY{l+s+s2}{\PYZdq{}}\PY{p}{,} \PY{n}{t}\PY{p}{)}\PY{p}{)}

\PY{c+c1}{\PYZsh{} Replace de apostrofes por traços para conservar palavras}
\PY{c+c1}{\PYZsh{} como don\PYZsq{}t e i\PYZsq{}m na hora de tokenizar}
\PY{n}{data}\PY{p}{[}\PY{l+s+s2}{\PYZdq{}}\PY{l+s+s2}{Text}\PY{l+s+s2}{\PYZdq{}}\PY{p}{]} \PY{o}{=} \PY{n}{data}\PY{p}{[}\PY{l+s+s2}{\PYZdq{}}\PY{l+s+s2}{Text}\PY{l+s+s2}{\PYZdq{}}\PY{p}{]}\PY{o}{.}\PY{n}{apply}\PY{p}{(}\PY{k}{lambda} \PY{n}{t}\PY{p}{:} \PY{n}{t}\PY{o}{.}\PY{n}{replace}\PY{p}{(}\PY{l+s+s2}{\PYZdq{}}\PY{l+s+se}{\PYZbs{}\PYZsq{}}\PY{l+s+s2}{\PYZdq{}}\PY{p}{,} \PY{l+s+s2}{\PYZdq{}}\PY{l+s+s2}{\PYZhy{}}\PY{l+s+s2}{\PYZdq{}}\PY{p}{)}\PY{p}{)}
\end{Verbatim}


Após este processamento inicial, o dataframe possui frases que estão quase prontas. 
No entanto, como mencionado, a maior parte são artigos e textos de blogs, que possuem mais de uma frase.

Para separar as frases, utilizou-se um split usando o ponto final como caractere para definir onde uma frase acaba.
Como a função split gera uma lista com as strings separadas, transformou-se esta lista em novas linhas do dataframe utilizando a função explode do Pandas.



\begin{Verbatim}[commandchars=\\\{\},frame=single,fontsize=\small, xleftmargin=0.5em]
\PY{c+c1}{\PYZsh{} Fazendo o split das frases por .}
\PY{n}{data}\PY{p}{[}\PY{l+s+s2}{\PYZdq{}}\PY{l+s+s2}{Text}\PY{l+s+s2}{\PYZdq{}}\PY{p}{]} \PY{o}{=} \PY{n}{data}\PY{p}{[}\PY{l+s+s2}{\PYZdq{}}\PY{l+s+s2}{Text}\PY{l+s+s2}{\PYZdq{}}\PY{p}{]}\PY{o}{.}\PY{n}{apply}\PY{p}{(}\PY{k}{lambda} \PY{n}{t}\PY{p}{:} \PY{n}{t}\PY{o}{.}\PY{n}{split}\PY{p}{(}\PY{l+s+s2}{\PYZdq{}}\PY{l+s+s2}{.}\PY{l+s+s2}{\PYZdq{}}\PY{p}{)}\PY{p}{)}

\PY{c+c1}{\PYZsh{} Comando explode para separar as listas em novas linhas do dataframe}
\PY{n}{data} \PY{o}{=} \PY{n}{data}\PY{o}{.}\PY{n}{explode}\PY{p}{(}\PY{l+s+s2}{\PYZdq{}}\PY{l+s+s2}{Text}\PY{l+s+s2}{\PYZdq{}}\PY{p}{)}
\end{Verbatim}


O dataframe agora possui apenas frases (e expandiu seu tamanho de 1 milhão de linhas para 3 milhões), pode-se continuar com a limpeza.
A etapa seguinte foi a remoção de espaços no início e no fim da frase, e também a remoção de espaços excessivos.



\begin{Verbatim}[commandchars=\\\{\},frame=single,fontsize=\small, xleftmargin=0.5em]
\PY{c+c1}{\PYZsh{} Removendo espaços em branco no início e no fim da frase}
\PY{n}{data}\PY{p}{[}\PY{l+s+s2}{\PYZdq{}}\PY{l+s+s2}{Text}\PY{l+s+s2}{\PYZdq{}}\PY{p}{]} \PY{o}{=} \PY{n}{data}\PY{p}{[}\PY{l+s+s2}{\PYZdq{}}\PY{l+s+s2}{Text}\PY{l+s+s2}{\PYZdq{}}\PY{p}{]}\PY{o}{.}\PY{n}{apply}\PY{p}{(}\PY{k}{lambda} \PY{n}{t}\PY{p}{:} \PY{n}{t}\PY{o}{.}\PY{n}{strip}\PY{p}{(}\PY{p}{)}\PY{p}{)}

\PY{c+c1}{\PYZsh{} Removendo tabs, newlines e espaços em branco em excesso}
\PY{n}{data}\PY{p}{[}\PY{l+s+s2}{\PYZdq{}}\PY{l+s+s2}{Text}\PY{l+s+s2}{\PYZdq{}}\PY{p}{]} \PY{o}{=} \PY{n}{data}\PY{p}{[}\PY{l+s+s2}{\PYZdq{}}\PY{l+s+s2}{Text}\PY{l+s+s2}{\PYZdq{}}\PY{p}{]}\PY{o}{.}\PY{n}{apply}\PY{p}{(}\PY{k}{lambda} \PY{n}{t}\PY{p}{:} \PY{n}{t}\PY{o}{.}\PY{n}{replace}\PY{p}{(}\PY{l+s+s2}{\PYZdq{}}\PY{l+s+s2}{/}\PY{l+s+s2}{\PYZbs{}}\PY{l+s+s2}{s}\PY{l+s+s2}{\PYZbs{}}\PY{l+s+s2}{s+/g}\PY{l+s+s2}{\PYZdq{}}\PY{p}{,} \PY{l+s+s2}{\PYZdq{}}\PY{l+s+s2}{ }\PY{l+s+s2}{\PYZdq{}}\PY{p}{)}\PY{p}{)}
\end{Verbatim}


A última etapa de limpeza foi a remoção de frases que possuem números no meio.
Optou-se por remover completamente essas frases, pois elas podem ser incoerentes caso sejam removidos apenas os números (por exemplo: "it costs 20 dollars" viraria "it costs dollars", o que pode não fazer muito sentido).
No caso das URLs optou-se por apenas remover as URLs, pois há menos chances de ter problema de coerência.



\begin{Verbatim}[commandchars=\\\{\},frame=single,fontsize=\small, xleftmargin=0.5em]
\PY{c+c1}{\PYZsh{} Removendo frases que contém números}
\PY{k}{def} \PY{n+nf}{hasNumbers}\PY{p}{(}\PY{n}{text}\PY{p}{)}\PY{p}{:}
    \PY{k}{return} \PY{n+nb}{any}\PY{p}{(}\PY{n}{char}\PY{o}{.}\PY{n}{isdigit}\PY{p}{(}\PY{p}{)} \PY{k}{for} \PY{n}{char} \PY{o+ow}{in} \PY{n}{text}\PY{p}{)}

\PY{n}{data}\PY{p}{[}\PY{l+s+s2}{\PYZdq{}}\PY{l+s+s2}{hasNumbers}\PY{l+s+s2}{\PYZdq{}}\PY{p}{]} \PY{o}{=} \PY{n}{data}\PY{p}{[}\PY{l+s+s2}{\PYZdq{}}\PY{l+s+s2}{Text}\PY{l+s+s2}{\PYZdq{}}\PY{p}{]}\PY{o}{.}\PY{n}{apply}\PY{p}{(}\PY{k}{lambda} \PY{n}{t}\PY{p}{:} \PY{n}{hasNumbers}\PY{p}{(}\PY{n}{t}\PY{p}{)}\PY{p}{)}
\PY{n}{data} \PY{o}{=} \PY{n}{data}\PY{p}{[}\PY{n}{data}\PY{p}{[}\PY{l+s+s2}{\PYZdq{}}\PY{l+s+s2}{hasNumbers}\PY{l+s+s2}{\PYZdq{}}\PY{p}{]} \PY{o}{==} \PY{k+kc}{False}\PY{p}{]}
\end{Verbatim}


Por fim, removeu-se as frases vazias, filtrou-se as frases que possuem entre 7 e 12 palavras (de modo a não ter frases nem muito pequenas e nem muito grandes), e transformou-se as frases para lowercase.



\begin{Verbatim}[commandchars=\\\{\},frame=single,fontsize=\small, xleftmargin=0.5em]
\PY{c+c1}{\PYZsh{} Removendo frases vazias}
\PY{n}{data} \PY{o}{=} \PY{n}{data}\PY{p}{[}\PY{n}{data}\PY{p}{[}\PY{l+s+s2}{\PYZdq{}}\PY{l+s+s2}{Text}\PY{l+s+s2}{\PYZdq{}}\PY{p}{]} \PY{o}{!=} \PY{l+s+s2}{\PYZdq{}}\PY{l+s+s2}{\PYZdq{}}\PY{p}{]}

\PY{c+c1}{\PYZsh{} Mantendo frases que possuem entre 7 e 12 palavras}
\PY{n}{data} \PY{o}{=} \PY{n}{data}\PY{p}{[}\PY{n}{data}\PY{p}{[}\PY{l+s+s2}{\PYZdq{}}\PY{l+s+s2}{Text}\PY{l+s+s2}{\PYZdq{}}\PY{p}{]}\PY{o}{.}\PY{n}{str}\PY{o}{.}\PY{n}{split}\PY{p}{(}\PY{l+s+s2}{\PYZdq{}}\PY{l+s+s2}{ }\PY{l+s+s2}{\PYZdq{}}\PY{p}{)}\PY{o}{.}\PY{n}{str}\PY{o}{.}\PY{n}{len}\PY{p}{(}\PY{p}{)} \PY{o}{\PYZgt{}}\PY{o}{=} \PY{l+m+mi}{7}\PY{p}{]}
\PY{n}{data} \PY{o}{=} \PY{n}{data}\PY{p}{[}\PY{n}{data}\PY{p}{[}\PY{l+s+s2}{\PYZdq{}}\PY{l+s+s2}{Text}\PY{l+s+s2}{\PYZdq{}}\PY{p}{]}\PY{o}{.}\PY{n}{str}\PY{o}{.}\PY{n}{split}\PY{p}{(}\PY{l+s+s2}{\PYZdq{}}\PY{l+s+s2}{ }\PY{l+s+s2}{\PYZdq{}}\PY{p}{)}\PY{o}{.}\PY{n}{str}\PY{o}{.}\PY{n}{len}\PY{p}{(}\PY{p}{)} \PY{o}{\PYZlt{}}\PY{o}{=} \PY{l+m+mi}{12}\PY{p}{]}

\PY{c+c1}{\PYZsh{} Lowercase}
\PY{n}{data}\PY{p}{[}\PY{l+s+s2}{\PYZdq{}}\PY{l+s+s2}{Text}\PY{l+s+s2}{\PYZdq{}}\PY{p}{]} \PY{o}{=} \PY{n}{data}\PY{p}{[}\PY{l+s+s2}{\PYZdq{}}\PY{l+s+s2}{Text}\PY{l+s+s2}{\PYZdq{}}\PY{p}{]}\PY{o}{.}\PY{n}{apply}\PY{p}{(}\PY{k}{lambda} \PY{n}{t}\PY{p}{:} \PY{n}{t}\PY{o}{.}\PY{n}{lower}\PY{p}{(}\PY{p}{)}\PY{p}{)}

\PY{c+c1}{\PYZsh{} Reset index}
\PY{n}{data}\PY{o}{.}\PY{n}{reset\PYZus{}index}\PY{p}{(}\PY{n}{inplace}\PY{o}{=}\PY{k+kc}{True}\PY{p}{)}

\PY{c+c1}{\PYZsh{} Transforming to list}
\PY{n}{sentences} \PY{o}{=} \PY{n}{data}\PY{p}{[}\PY{l+s+s2}{\PYZdq{}}\PY{l+s+s2}{Text}\PY{l+s+s2}{\PYZdq{}}\PY{p}{]}\PY{o}{.}\PY{n}{tolist}\PY{p}{(}\PY{p}{)}
\end{Verbatim}


Com as frases disponíveis, pode-se prosseguir para a construção da ferramenta.

\section{Montando a ferramenta}

\subsection{N-gramas}

Para construir a ferramenta será utilizada um modelo de N-gramas.

Um N-grama é um conceito utilizado em linguística computacional e probabilidade, e representa um conjunto de itens de um determinado texto ou fala, podendo ser sílabas, fonemas, palavras, por exemplo. 
Um N-grama de dois itens é chamado de bigrama, um de três itens é chamado de trigrama, etc.
	
Os N-gramas podem possuir algumas aplicações diferentes, e são muito utilizados nas áreas de:

\begin{itemize}
    \item probabilidade;
    \item teoria da comunicação;
    \item linguística computacional (processamento estatístico de linguagem natural);
    \item biologia computacional;
    \item compressão de dados.
\end{itemize}

\subsubsection{Modelo de predição}

De modo a construir nosso preditor, utilizará-se um modelo que segue as seguintes etapas:

\begin{enumerate}
    \item gera-se todos os N-gramas de 2 a 4 itens das frases de nosso dataframe;
    \item calcula-se as probabilidades de cada N-grama estar no texto;
    \item gera-se os N-gramas da nossa frase de entrada de acordo com o modelo escolhido;
    \item encontra-se os primeiros N-gramas mais prováveis de serem o N-grama buscado;
    \item pega-se a última palavra deles para preencher a frase de entrada.
\end{enumerate}

Contudo, há palavras que são extremamente frequentes na língua inglesa (e em outras línguas).
Essas são chamadas de stopwords, e para nosso modelo, os N-gramas compostos exclusivamente por stopwords serão descartados.
A biblioteca NLTK possui os stopwords da língua inglesa disponível para esta remoção.

A maior vantagem do modelo de N-gramas é a sua simplicidade de entendimento e implementação.
No entanto, sua maior desvantagem é que ele pode nem sempre acertar o que estava sendo buscado.

O código abaixo mostra o desenvolvimento da etapa 1 do modelo.



\begin{Verbatim}[commandchars=\\\{\},frame=single,fontsize=\small, xleftmargin=0.5em]
\PY{c+c1}{\PYZsh{} Imports}
\PY{k+kn}{from} \PY{n+nn}{nltk} \PY{k+kn}{import} \PY{n}{word\PYZus{}tokenize}
\PY{k+kn}{from} \PY{n+nn}{nltk}\PY{n+nn}{.}\PY{n+nn}{util} \PY{k+kn}{import} \PY{n}{ngrams}
\PY{k+kn}{from} \PY{n+nn}{nltk}\PY{n+nn}{.}\PY{n+nn}{corpus} \PY{k+kn}{import} \PY{n}{stopwords}

\PY{c+c1}{\PYZsh{} Lista dos N\PYZhy{}gramas}
\PY{n}{bigrams} \PY{o}{=} \PY{p}{[}\PY{p}{]}
\PY{n}{trigrams} \PY{o}{=} \PY{p}{[}\PY{p}{]}
\PY{n}{fourgrams} \PY{o}{=} \PY{p}{[}\PY{p}{]}

\PY{c+c1}{\PYZsh{} Stopwords da língua inglesa}
\PY{n}{stop\PYZus{}words} \PY{o}{=} \PY{n+nb}{set}\PY{p}{(}\PY{n}{stopwords}\PY{o}{.}\PY{n}{words}\PY{p}{(}\PY{l+s+s2}{\PYZdq{}}\PY{l+s+s2}{english}\PY{l+s+s2}{\PYZdq{}}\PY{p}{)}\PY{p}{)}

\PY{c+c1}{\PYZsh{} Iterando sobre cada frase}
\PY{k}{for} \PY{n}{sentence} \PY{o+ow}{in} \PY{n}{sentences}\PY{p}{:}
    \PY{c+c1}{\PYZsh{} Tokenizando as frases}
    \PY{n}{words} \PY{o}{=} \PY{n}{word\PYZus{}tokenize}\PY{p}{(}\PY{n}{sentence}\PY{p}{)}
    
    \PY{c+c1}{\PYZsh{} Extendendo a lista dos N\PYZhy{}gramas com as palavras}
    \PY{n}{bigrams}\PY{o}{.}\PY{n}{extend}\PY{p}{(}\PY{n+nb}{list}\PY{p}{(}\PY{n}{ngrams}\PY{p}{(}\PY{n}{words}\PY{p}{,} \PY{l+m+mi}{2}\PY{p}{)}\PY{p}{)}\PY{p}{)}
    \PY{n}{trigrams}\PY{o}{.}\PY{n}{extend}\PY{p}{(}\PY{n+nb}{list}\PY{p}{(}\PY{n}{ngrams}\PY{p}{(}\PY{n}{words}\PY{p}{,} \PY{l+m+mi}{3}\PY{p}{)}\PY{p}{)}\PY{p}{)}
    \PY{n}{fourgrams}\PY{o}{.}\PY{n}{extend}\PY{p}{(}\PY{n+nb}{list}\PY{p}{(}\PY{n}{ngrams}\PY{p}{(}\PY{n}{words}\PY{p}{,} \PY{l+m+mi}{4}\PY{p}{)}\PY{p}{)}\PY{p}{)}   

\PY{c+c1}{\PYZsh{} Remove N\PYZhy{}gramas que são compostos exclusivamente por stopwords}
\PY{k}{def} \PY{n+nf}{remove\PYZus{}stopwords}\PY{p}{(}\PY{n}{ngram}\PY{p}{:} \PY{n+nb}{list}\PY{p}{)}\PY{p}{:}
    \PY{n}{new\PYZus{}ngram} \PY{o}{=} \PY{p}{[}\PY{p}{]}

    \PY{k}{for} \PY{n}{sequence} \PY{o+ow}{in} \PY{n}{ngram}\PY{p}{:}
        \PY{c+c1}{\PYZsh{} Assume\PYZhy{}se que o N\PYZhy{}grama é composto por stopwords}
        \PY{n}{count} \PY{o}{=} \PY{l+m+mi}{0}

        \PY{c+c1}{\PYZsh{} Percorrendo cada palavra do N\PYZhy{}grama}
        \PY{k}{for} \PY{n}{word} \PY{o+ow}{in} \PY{n}{sequence}\PY{p}{:}
            \PY{c+c1}{\PYZsh{} Se o N\PYZhy{}grama não tem nenhuma palavra que é uma stopword}
            \PY{c+c1}{\PYZsh{} o count vira 1; se não, se mantém em zero}
            \PY{n}{count} \PY{o}{=} \PY{n}{count} \PY{o+ow}{or} \PY{l+m+mi}{0} \PY{k}{if} \PY{n}{word} \PY{o+ow}{in} \PY{n}{stop\PYZus{}words} \PY{k}{else} \PY{n}{count} \PY{o+ow}{or} \PY{l+m+mi}{1}

        \PY{c+c1}{\PYZsh{} Adiciona\PYZhy{}se o N\PYZhy{}grama se ele não for composto por stopwords}
        \PY{k}{if} \PY{n}{count} \PY{o}{==} \PY{l+m+mi}{1}\PY{p}{:}
            \PY{n}{new\PYZus{}ngram}\PY{o}{.}\PY{n}{append}\PY{p}{(}\PY{n}{sequence}\PY{p}{)}
            
    \PY{k}{return} \PY{n}{new\PYZus{}ngram}

\PY{n}{bigrams} \PY{o}{=} \PY{n}{remove\PYZus{}stopwords}\PY{p}{(}\PY{n}{bigrams}\PY{p}{)}
\PY{n}{trigrams} \PY{o}{=} \PY{n}{remove\PYZus{}stopwords}\PY{p}{(}\PY{n}{trigrams}\PY{p}{)}
\PY{n}{fourgrams} \PY{o}{=} \PY{n}{remove\PYZus{}stopwords}\PY{p}{(}\PY{n}{fourgrams}\PY{p}{)}
\end{Verbatim}


\subsection{Probabilidades}

A fórmula para calcular a probabilidade de um N-grama estar presente em um texto é dada por:

$P(N) = \frac{\text{ocorrencias(N-grama)}}{\text{total(N-gramas)}}$

Mas e se não houver ocorrências do N-grama no texto? A probabilidade neste caso seria zero.
Porém, de modo a evitar estes casos, utiliza-se uma técnica de smoothing chamada Add-1 Smoothing (também conhecida como Laplace Smoothing).

Esta técnica consiste em adicionar 1 ao número de ocorrências de todos os N-gramas (ou seja, adicionar 1 ao númerador da fórmula anterior).
Para compensar esta adição, devemos contar novamente quantos N-gramas temos, mas há uma forma prática de realizar isto: basta somar a quantidade de N-gramas únicos ao denominador da fórmula.

Assim, a nova probabilidade de um N-grama estar presente no texto é dada por:

$P(N) = \frac{\text{ocorrencias(N-grama)} + 1}{\text{total(N-gramas)} + \text{total(N-gramas unicos)}}$

O código abaixo mostra o desenvolvimento da etapa 2 do modelo.



\begin{Verbatim}[commandchars=\\\{\},frame=single,fontsize=\small, xleftmargin=0.5em]
\PY{c+c1}{\PYZsh{} Utiliza\PYZhy{}se o módulo counter para obter as contagens de cada N\PYZhy{}grama}
\PY{k+kn}{from} \PY{n+nn}{collections} \PY{k+kn}{import} \PY{n}{Counter}
\PY{n}{n2}\PY{p}{,} \PY{n}{n3}\PY{p}{,} \PY{n}{n4} \PY{o}{=} \PY{n}{Counter}\PY{p}{(}\PY{n}{bigrams}\PY{p}{)}\PY{p}{,} \PY{n}{Counter}\PY{p}{(}\PY{n}{trigrams}\PY{p}{)}\PY{p}{,} \PY{n}{Counter}\PY{p}{(}\PY{n}{fourgrams}\PY{p}{)}

\PY{c+c1}{\PYZsh{} Gera\PYZhy{}se os casos únicos para os bigramas, os trigramas e os quadrigramas}
\PY{n}{s2}\PY{p}{,} \PY{n}{s3}\PY{p}{,} \PY{n}{s4} \PY{o}{=} \PY{n+nb}{set}\PY{p}{(}\PY{n}{bigrams}\PY{p}{)}\PY{p}{,} \PY{n+nb}{set}\PY{p}{(}\PY{n}{trigrams}\PY{p}{)}\PY{p}{,} \PY{n+nb}{set}\PY{p}{(}\PY{n}{fourgrams}\PY{p}{)}

\PY{c+c1}{\PYZsh{} Utilizando a fórmula de probabilidades}
\PY{n}{p2} \PY{o}{=} \PY{p}{[}\PY{p}{(}\PY{n}{n}\PY{p}{,} \PY{p}{(}\PY{n}{n2}\PY{p}{[}\PY{n}{n}\PY{p}{]} \PY{o}{+} \PY{l+m+mi}{1}\PY{p}{)}\PY{o}{/}\PY{p}{(}\PY{n+nb}{len}\PY{p}{(}\PY{n}{n2}\PY{p}{)} \PY{o}{+} \PY{n+nb}{len}\PY{p}{(}\PY{n}{s2}\PY{p}{)}\PY{p}{)}\PY{p}{)} \PY{k}{for} \PY{n}{n} \PY{o+ow}{in} \PY{n}{s2}\PY{p}{]}
\PY{n}{p3} \PY{o}{=} \PY{p}{[}\PY{p}{(}\PY{n}{n}\PY{p}{,} \PY{p}{(}\PY{n}{n3}\PY{p}{[}\PY{n}{n}\PY{p}{]} \PY{o}{+} \PY{l+m+mi}{1}\PY{p}{)}\PY{o}{/}\PY{p}{(}\PY{n+nb}{len}\PY{p}{(}\PY{n}{n3}\PY{p}{)} \PY{o}{+} \PY{n+nb}{len}\PY{p}{(}\PY{n}{s3}\PY{p}{)}\PY{p}{)}\PY{p}{)} \PY{k}{for} \PY{n}{n} \PY{o+ow}{in} \PY{n}{s3}\PY{p}{]}
\PY{n}{p4} \PY{o}{=} \PY{p}{[}\PY{p}{(}\PY{n}{n}\PY{p}{,} \PY{p}{(}\PY{n}{n4}\PY{p}{[}\PY{n}{n}\PY{p}{]} \PY{o}{+} \PY{l+m+mi}{1}\PY{p}{)}\PY{o}{/}\PY{p}{(}\PY{n+nb}{len}\PY{p}{(}\PY{n}{n4}\PY{p}{)} \PY{o}{+} \PY{n+nb}{len}\PY{p}{(}\PY{n}{s4}\PY{p}{)}\PY{p}{)}\PY{p}{)} \PY{k}{for} \PY{n}{n} \PY{o+ow}{in} \PY{n}{s4}\PY{p}{]}

\PY{c+c1}{\PYZsh{} Ordena\PYZhy{}se as listas de acordo com a probabilidade de cada N\PYZhy{}grama}
\PY{c+c1}{\PYZsh{} em ordem descrescente}
\PY{n}{p2} \PY{o}{=} \PY{n+nb}{sorted}\PY{p}{(}\PY{n}{p2}\PY{p}{,} \PY{n}{key}\PY{o}{=}\PY{k}{lambda} \PY{n}{w}\PY{p}{:} \PY{n}{w}\PY{p}{[}\PY{l+m+mi}{1}\PY{p}{]}\PY{p}{,} \PY{n}{reverse}\PY{o}{=}\PY{k+kc}{True}\PY{p}{)}
\PY{n}{p3} \PY{o}{=} \PY{n+nb}{sorted}\PY{p}{(}\PY{n}{p3}\PY{p}{,} \PY{n}{key}\PY{o}{=}\PY{k}{lambda} \PY{n}{w}\PY{p}{:} \PY{n}{w}\PY{p}{[}\PY{l+m+mi}{1}\PY{p}{]}\PY{p}{,} \PY{n}{reverse}\PY{o}{=}\PY{k+kc}{True}\PY{p}{)}
\PY{n}{p4} \PY{o}{=} \PY{n+nb}{sorted}\PY{p}{(}\PY{n}{p4}\PY{p}{,} \PY{n}{key}\PY{o}{=}\PY{k}{lambda} \PY{n}{w}\PY{p}{:} \PY{n}{w}\PY{p}{[}\PY{l+m+mi}{1}\PY{p}{]}\PY{p}{,} \PY{n}{reverse}\PY{o}{=}\PY{k+kc}{True}\PY{p}{)}
\end{Verbatim}


\subsection{Gerando a predição}

Para realizar a predição da próxima palavra a ser digitada em um texto, criou-se uma função que recebe como entrada:
\begin{itemize}
    \item uma string;
    \item um inteiro n que define as n primeiras palavras mais prováveis a serem retornadas pelo preditor;
    \item um inteiro model que define o modelo de N-gramas a ser utilizado (que por padrão utiliza os bigramas);
\end{itemize}

O bloco abaixo mostra o código da função.



\begin{Verbatim}[commandchars=\\\{\},frame=single,fontsize=\small, xleftmargin=0.5em]
\PY{k}{def} \PY{n+nf}{return\PYZus{}prediction}\PY{p}{(}\PY{n}{inp}\PY{p}{,} \PY{n}{n}\PY{p}{,} \PY{n}{model}\PY{o}{=}\PY{l+m+mi}{2}\PY{p}{)}\PY{p}{:}
    \PY{c+c1}{\PYZsh{} Lista de possiveis palavras}
    \PY{n}{pred} \PY{o}{=} \PY{p}{[}\PY{p}{]}
    
    \PY{c+c1}{\PYZsh{} Tokenizando o input}
    \PY{n}{inp} \PY{o}{=} \PY{n}{word\PYZus{}tokenize}\PY{p}{(}\PY{n}{inp}\PY{p}{)}
    
    \PY{c+c1}{\PYZsh{} Modelo escolhido}
    \PY{n}{p} \PY{o}{=} \PY{k+kc}{None}
    
    \PY{k}{if} \PY{n}{model} \PY{o}{==} \PY{l+m+mi}{2}\PY{p}{:}
        \PY{n}{p} \PY{o}{=} \PY{n}{p2}
        
    \PY{k}{elif} \PY{n}{model} \PY{o}{==} \PY{l+m+mi}{3}\PY{p}{:}
        \PY{n}{p} \PY{o}{=} \PY{n}{p3}
                
    \PY{k}{elif} \PY{n}{model} \PY{o}{==} \PY{l+m+mi}{4}\PY{p}{:}
        \PY{n}{p} \PY{o}{=} \PY{n}{p4}
    
    \PY{c+c1}{\PYZsh{} Se a quantidade de palavras do input for menor que o tamanho do modelo }
    \PY{c+c1}{\PYZsh{} escolhido não é possível formar um N\PYZhy{}grama.}
    \PY{c+c1}{\PYZsh{} Logo, devemos retornar um erro.}
    \PY{k}{if} \PY{n+nb}{len}\PY{p}{(}\PY{n}{inp}\PY{p}{)} \PY{o}{\PYZlt{}} \PY{n}{model} \PY{o}{\PYZhy{}} \PY{l+m+mi}{1}\PY{p}{:}
        \PY{k}{return} \PY{l+s+sa}{f}\PY{l+s+s2}{\PYZdq{}}\PY{l+s+s2}{Could not form a n\PYZhy{}gram of size }\PY{l+s+si}{\PYZob{}}\PY{n}{model}\PY{l+s+si}{\PYZcb{}}\PY{l+s+s2}{ with given input.}\PY{l+s+s2}{\PYZdq{}}

    \PY{c+c1}{\PYZsh{} Forma\PYZhy{}se os N\PYZhy{}gramas de tamanho model \PYZhy{} 1}
    \PY{n}{in\PYZus{}ngram} \PY{o}{=} \PY{n+nb}{list}\PY{p}{(}\PY{n}{ngrams}\PY{p}{(}\PY{n}{inp}\PY{p}{,} \PY{n}{model} \PY{o}{\PYZhy{}} \PY{l+m+mi}{1}\PY{p}{)}\PY{p}{)}

    \PY{c+c1}{\PYZsh{} Pega\PYZhy{}se o último dos N\PYZhy{}gramas do input, pois nosso modelo irá buscar }
    \PY{c+c1}{\PYZsh{} o N\PYZhy{}grama mais provável que encaixa para ele}

    \PY{c+c1}{\PYZsh{} Por exemplo: se a entrada é \PYZdq{}i want to\PYZdq{} e o nosso modelo é }
    \PY{c+c1}{\PYZsh{} de tamanho 2, utilizaremos o N\PYZhy{}grama (\PYZdq{}to\PYZdq{})}
    \PY{n}{in\PYZus{}ngram} \PY{o}{=} \PY{n}{in\PYZus{}ngram}\PY{p}{[}\PY{o}{\PYZhy{}}\PY{l+m+mi}{1}\PY{p}{]}
    
    \PY{c+c1}{\PYZsh{} Quantidade de palavras que podem encaixar na frase}
    \PY{n}{count} \PY{o}{=} \PY{l+m+mi}{0}
    
    \PY{c+c1}{\PYZsh{} Iterando sobre cada uma das tuplas (N\PYZhy{}grama, probabilidade) do modelo}
    \PY{k}{for} \PY{n}{wp} \PY{o+ow}{in} \PY{n}{p}\PY{p}{:}
        \PY{c+c1}{\PYZsh{} Seleciona\PYZhy{}se apenas o N\PYZhy{}grama}
        \PY{n}{p\PYZus{}ngram} \PY{o}{=} \PY{n}{wp}\PY{p}{[}\PY{l+m+mi}{0}\PY{p}{]}

        \PY{c+c1}{\PYZsh{} Remove\PYZhy{}se a última palavra do N\PYZhy{}grama, de modo a obter um N\PYZhy{}grama }
        \PY{c+c1}{\PYZsh{} de tamanho igual ao N\PYZhy{}grama gerado para nossa entrada}
        \PY{n}{model\PYZus{}ngram} \PY{o}{=} \PY{n}{p\PYZus{}ngram}\PY{p}{[}\PY{p}{:}\PY{o}{\PYZhy{}}\PY{l+m+mi}{1}\PY{p}{]}

        \PY{c+c1}{\PYZsh{} Voltando ao exemplo anterior, podemos encontrar um N\PYZhy{}grama }
        \PY{c+c1}{\PYZsh{} do tipo (\PYZdq{}to\PYZdq{}, \PYZdq{}do\PYZdq{}). Removendo\PYZhy{}se a última palavra, }
        \PY{c+c1}{\PYZsh{} teremos um N\PYZhy{}grama (\PYZdq{}to\PYZdq{}), que é igual ao nosso }
        \PY{c+c1}{\PYZsh{} N\PYZhy{}grama da entrada}

        \PY{c+c1}{\PYZsh{} Se o N\PYZhy{}grama do modelo é igual ao N\PYZhy{}grama da entrada}
        \PY{c+c1}{\PYZsh{} tivemos um match}
        \PY{k}{if} \PY{n}{model\PYZus{}ngram} \PY{o}{==} \PY{n}{in\PYZus{}ngram}\PY{p}{:}
            \PY{n}{count} \PY{o}{+}\PY{o}{=} \PY{l+m+mi}{1}
            
            \PY{c+c1}{\PYZsh{} Seleciona\PYZhy{}se apenas a última palavra do N\PYZhy{}grama do modelo}
            \PY{c+c1}{\PYZsh{} e faz\PYZhy{}se o replace do hífen artificial pela apóstrofe}
            \PY{n}{match} \PY{o}{=} \PY{n}{p\PYZus{}ngram}\PY{p}{[}\PY{o}{\PYZhy{}}\PY{l+m+mi}{1}\PY{p}{]}\PY{o}{.}\PY{n}{replace}\PY{p}{(}\PY{l+s+s2}{\PYZdq{}}\PY{l+s+s2}{\PYZhy{}}\PY{l+s+s2}{\PYZdq{}}\PY{p}{,} \PY{l+s+s2}{\PYZdq{}}\PY{l+s+s2}{\PYZsq{}}\PY{l+s+s2}{\PYZdq{}}\PY{p}{)}
            \PY{n}{pred}\PY{o}{.}\PY{n}{append}\PY{p}{(}\PY{n}{match}\PY{p}{)}
            
            \PY{c+c1}{\PYZsh{} Novamente utilizando o exemplo anterior, como tivemos um }
            \PY{c+c1}{\PYZsh{} match, o modelo assume que queriamos o N\PYZhy{}grama }
            \PY{c+c1}{\PYZsh{} (\PYZdq{}to\PYZdq{}, \PYZdq{}do\PYZdq{}); logo, preenche a frase }
            \PY{c+c1}{\PYZsh{} \PYZdq{}i want to\PYZdq{} com a palavra \PYZdq{}do\PYZdq{}}

            \PY{c+c1}{\PYZsh{} Se obtivemos o número exigido de palavras, quebra o loop}
            \PY{k}{if} \PY{n}{count} \PY{o}{==} \PY{n}{n}\PY{p}{:}
                \PY{k}{break}
                
    \PY{c+c1}{\PYZsh{} Se não conseguimos obter palavras suficentes}
    \PY{c+c1}{\PYZsh{} preenche\PYZhy{}se a lista com N/A}
    \PY{k}{if} \PY{n}{count} \PY{o}{\PYZlt{}} \PY{n}{n}\PY{p}{:}
        \PY{n}{pred} \PY{o}{+}\PY{o}{=} \PY{p}{[}\PY{l+s+s2}{\PYZdq{}}\PY{l+s+s2}{N/A}\PY{l+s+s2}{\PYZdq{}}\PY{p}{]} \PY{o}{*} \PY{p}{(}\PY{n}{n} \PY{o}{\PYZhy{}} \PY{n}{count}\PY{p}{)}
    
    \PY{k}{return} \PY{n}{pred}
\end{Verbatim}


\section{Testando o modelo}

Para testar o modelo, utilizou-se a frase "i want to" (em letras minúsculas pois os dados estão em lowercase), escolheu-se obter as 5 primeiras palavras, e escolheu-se o modelo de trigramas.



\begin{Verbatim}[commandchars=\\\{\},frame=single,fontsize=\small, xleftmargin=0.5em]
\PY{n}{example} \PY{o}{=} \PY{l+s+s2}{\PYZdq{}}\PY{l+s+s2}{i want to}\PY{l+s+s2}{\PYZdq{}}
\PY{n}{pred} \PY{o}{=} \PY{n}{return\PYZus{}prediction}\PY{p}{(}\PY{n}{example}\PY{p}{,} \PY{n}{n}\PY{o}{=}\PY{l+m+mi}{5}\PY{p}{,} \PY{n}{model}\PY{o}{=}\PY{l+m+mi}{3}\PY{p}{)}

\PY{k}{for} \PY{n}{word} \PY{o+ow}{in} \PY{n}{pred}\PY{p}{:}
    \PY{n+nb}{print}\PY{p}{(}\PY{n}{example} \PY{o}{+} \PY{l+s+s2}{\PYZdq{}}\PY{l+s+s2}{ }\PY{l+s+s2}{\PYZdq{}} \PY{o}{+} \PY{n}{word}\PY{p}{)}
\end{Verbatim}

\begin{Verbatim}[commandchars=\\\{\},frame=leftline,fontsize=\small, xleftmargin=0.5em]
i want to be
i want to see
i want to do
i want to go
i want to make
\end{Verbatim}


Como mencionado anteriormente porém, o modelo nem sempre é preciso. O exemplo abaixo utiliza a frase "let me see your" e um modelo de quadrigramas.



\begin{Verbatim}[commandchars=\\\{\},frame=single,fontsize=\small, xleftmargin=0.5em]
\PY{n}{example} \PY{o}{=} \PY{l+s+s2}{\PYZdq{}}\PY{l+s+s2}{let me see your}\PY{l+s+s2}{\PYZdq{}}
\PY{n}{pred} \PY{o}{=} \PY{n}{return\PYZus{}prediction}\PY{p}{(}\PY{n}{example}\PY{p}{,} \PY{n}{n}\PY{o}{=}\PY{l+m+mi}{5}\PY{p}{,} \PY{n}{model}\PY{o}{=}\PY{l+m+mi}{4}\PY{p}{)}

\PY{k}{for} \PY{n}{word} \PY{o+ow}{in} \PY{n}{pred}\PY{p}{:}
    \PY{n+nb}{print}\PY{p}{(}\PY{n}{example} \PY{o}{+} \PY{l+s+s2}{\PYZdq{}}\PY{l+s+s2}{ }\PY{l+s+s2}{\PYZdq{}} \PY{o}{+} \PY{n}{word}\PY{p}{)}
\end{Verbatim}

\begin{Verbatim}[commandchars=\\\{\},frame=leftline,fontsize=\small, xleftmargin=0.5em]
let me see your N/A
let me see your N/A
let me see your N/A
let me see your N/A
let me see your N/A
\end{Verbatim}


\section{Conclusão e iterações futuras}

É possível concluir que a partir de um modelo de N-gramas é possível fazer um preditor de palavras de texto eficaz e prático.
Todavia, o modelo não é muito preciso, especialmente para frases grandes e para modelos de quadrigramas. 
Outro problema é que o modelo não considera o contexto completo da frase, e seleciona-se apenas as últimas palavras desta para gerar a predição.

Para melhorar algumas destas falhas, pode-se considerar construir N-gramas maiores para gerar modelos mais precisos.
Além disso, poderia-se implementar um sistema de retroalimentação da base de dados, de modo a fornecer mais dados.

A ferramenta também poderia receber novas features. Um exemplo interessante seria o uso de um sistema de input em tempo real, de modo que o usuário tenha uma experiência mais suave e agradável.

Além disso, pode-se implementar o uso de outras linguagens no modelo. 
Para tal, seriam necessários ajustes no pré-processamento e limpeza dos dados, e também seria necessário obter as stopwords da nova língua.

\end{document}
