\documentclass[a4paper,11pt,final]{article}
\usepackage{fancyvrb, color, graphicx, hyperref, amsmath, url}
\usepackage{palatino}
\usepackage{pygments}
\usepackage[a4paper,text={16.5cm,25.2cm},centering]{geometry}

\hypersetup  
{   pdfauthor = {Bruno Domingues e Michel Moraes},
  pdftitle={Autocomplete},
  colorlinks=TRUE,
  linkcolor=black,
  citecolor=blue,
  urlcolor=blue
}

\setlength{\parindent}{0pt}
\setlength{\parskip}{1.2ex}


        
\title{Projeto Final de NLP - Autocomplete}
\author{Bruno Domingues e Michel Moraes}

\begin{document}
\maketitle

\section{Introdução}

O uso de abreviações e atalhos na hora de digitar se encontra cada vez mais presente no dia a dia: seja ao mandar mensagens, seja ao escrever textos ou até mesmo para pesquisar na internet.
A partir deste desejo de economizar tempo, surgiram diversas ferramentas capazes de completar frases e palavras, conhecidas geralmente como autocomplete e autosuggest.

Com base no aumento pela busca de ferramentas deste gênero, optou-se por fazer uma ferramenta que opera de forma similar ao Google: digita-se algumas palavras, e a ferramenta retorna quais as próximas palavras que podem completar a frase.



\end{document}



